\documentclass{article}\usepackage{amsmath,amssymb,amsthm,tikz,tkz-graph,color,chngpage,soul,hyperref,csquotes,graphicx,floatrow}\newcommand*{\QEDB}{\hfill\ensuremath{\square}}\newtheorem*{prop}{Proposition}\renewcommand{\theenumi}{\alph{enumi}}\usepackage[shortlabels]{enumitem}\usepackage[nobreak=true]{mdframed}\usetikzlibrary{matrix,calc}\MakeOuterQuote{"}\usepackage[margin=0.75in]{geometry} \newtheorem{theorem}{Theorem}
\newcommand{\dincludegraphics}{\includegraphics[width=0.5\textwidth]}
\newcommand{\tincludegraphics}{\includegraphics[width=0.33\textwidth]}

\title{EE16A - Midterm 2 Notes}
\author{Name: Felix Su$\quad$SID: 25794773}
\date{Spring 2016$\quad$GSI: Ena Hariyoshi}
\begin{document}
\maketitle

%%%% Topic %%%%
\subsection*{Superposition}
%%%% Notes %%%%
\begin{itemize}
    \item Investigate circuit using one source at a time
    \item Sum all components algebraically (keep in mind polarity)
\end{itemize}
%%%% Topic %%%%
\subsection*{Passive Sign Convention}
%%%% Notes %%%%
\begin{itemize}
    \item Consuming Power: Current enters ($V^+$); Supply Power: Current enters ($V^-$)
    \item Supply: Voltage/Current sources, discharging capacitors
    \item Consume: Resistors, charging capacitors
\end{itemize}
%%%% Topic %%%%
\subsection*{OpAmps}
%%%% Notes %%%%
\tincludegraphics{opsum}
\begin{itemize}
    \item $V_{out} = A(V^+ - V^-)$
    \item $V_{s-} \ge V_{out} \ge V_{s-}$
\end{itemize}
\begin{mdframed}
\textbf{Golden Rules:}
\begin{itemize}
    \item $I^- = I^+ = 0$
    \item $V^-=V^+$ (only with Neg FB)
\end{itemize}
\end{mdframed}
%%%% Topic %%%%
\subsection*{Charge Sharing}
%%%% Notes %%%%
\begin{itemize}
    \item For each phase: compute $Q_{tot}=\sum Q_n$ in terms of CV
    \item Equate $Q_{tot}$ of all phases: solve
    \item When getting $\Delta V$ for $Q_n=\frac{C_n}{\Delta V_n}$, subract ($V^+ -  - V^-$)
\end{itemize}
%%%% Topic %%%%
\subsection*{Capacitance}
%%%% Notes %%%%
\begin{itemize}
    \item $Q=CV, C=\varepsilon\frac{A}{d}, E=\frac{CV^2}{2}$
    \item Parallel: $C_p = \sum C_n$ V is same, but Q may not be
    \item Series: $C_s = \frac{C_1C_2}{C_1+C_2}, \frac{1}{C_s}=\sum\frac{1}{C_n}$
    \item Current applied to cap $\Rightarrow Q$ increases and V increases over time
    \item Dischargin cap $\Rightarrow V$ drops
    \item $Q$ on cap. after $t$ time = $It$
\end{itemize}
%%%% Topic %%%%
\subsection*{Thevenin/Norton}
%%%% Notes %%%%
\begin{itemize}
    \item Treat output terminals as open circuit ($V_{out}=V_{th}$)
    \item Treat output terminals as short circuit ($I_{sc}=I_{no}$
    \item $R_{th} = R_{no} = \frac{V_{th}}{I_{no}}$
\end{itemize}
%%%% Topic %%%%
\subsection*{Dividers}
%%%% Notes %%%%
\begin{itemize}
    \item Voltage: $V_{out} = V_{in}\frac{R_2}{R_1R_2}$
    \item Current: $I_{x} = I_t \frac{R_t}{R_x+R_t}, I_1=I_{tot}\frac{R_2}{R_1+R_2}$
\end{itemize}
%%%% Topic %%%%
\subsection*{KVL/VCL}
%%%% Notes %%%%
\begin{itemize}
    \item KVL: net potential around any loop = 0: $\sum V = 0$
    \item KCL: Charge is always conserved: $\sum I = 0$, $I_{in} = I_{out}$
\end{itemize}
%%%% Topic %%%%
\subsection*{Resistors}
%%%% Notes %%%%
\begin{itemize}
    \item Ohm's Law: $V=IR$, Resistivity: $R=\rho\frac{L}{A}$
    \item Series: $R_{tot} = \sum R_n$: Current equiv. Voltage splits proportionally
    \item Parallel: $R_{tot} = \frac{R_1R_2}{R_1+R_2}, \frac{1}{R_{tot}} = \sum\frac{1}{R_n}$: Voltage drop equiv. Current split proportionally
\end{itemize}
%%%% Topic %%%%
\subsection*{Null Space}
%%%% Notes %%%%
\begin{itemize}
    \item Full Row reduction
    \item Set free vars to arbitrary values (r,s,...)
    \item Solve $x$ in terms of free vars
    \item $dim(Nul(A)) =$ No. of free vars
\end{itemize}
\end{document}